% TeX шаблон, пример оформления отчёта по лабораторной работе.
% Автор: Шмаков И.А.
% Версия от: 05 ноября 2018 года.
% Сборка документа из командной строки:
% ~$ pdflatex -shell-escape main.tex

\documentclass[a4paper,14pt]{extarticle}
\usepackage[utf8]{inputenc}
\usepackage[T2A]{fontenc}
\usepackage[english,russian]{babel}

% одинарный интервал
\usepackage{setspace}
\singlespacing 

\usepackage[left=3cm, right=1cm, top=1.5cm, bottom=1.5cm]{geometry}
\usepackage{icomma} % "Умная" запятая: $0,2$ --- число, $0, 2$ --- перечисление
\usepackage{indentfirst} % Красная строка.

% для гиперссылок и выделение их цветом
\usepackage{xcolor}
\usepackage{hyperref}

 % Цвета для гиперссылок
\definecolor{linkcolor}{HTML}{000000} % цвет ссылок
\definecolor{urlcolor}{HTML}{799B03} % цвет гиперссылок
\hypersetup{pdfstartview=FitH,  linkcolor=linkcolor, urlcolor=urlcolor, colorlinks=true}

% Пакет отвечающий за листинги.
\usepackage[outputdir=build]{minted} 
\renewcommand\listingscaption{Листин}

% Фикс "red box" в minted 
\usemintedstyle{xcode}

% Подключение графики
\usepackage{graphicx}
\usepackage{float}
\DeclareGraphicsExtensions{.png}

\renewcommand{\thesection}{\arabic{section}}

\begin{document}
\begin{titlepage}
  \begin{center}
    \MakeUppercase{Министерство науки и высшего образования Российской Федерации} \\
    \MakeUppercase{ФГБОУ ВО Алтайский государственный университет}
    \vspace{0.25cm}
    
    Физико-технический факультет
    
    Кафедра вычислительной техники и электроники
    \vfill
    
    \textsc{Отчёт по лабораторной работе №1 по курсу \\ <<Практикум по ТРПО>>}
  \bigskip

\end{center}
\vfill

\newlength{\ML}
\settowidth{\ML}{«\underline{\hspace{0.6cm}}» \underline{\hspace{2cm}}}
\hfill\begin{minipage}{0.5\textwidth}
  Выполнил студент 4-го курса, \\ 576 группы:\\
  \underline{\hspace{\ML}} В.\,Е.~Щербаков\\
  «\underline{\hspace{0.7cm}}» \underline{\hspace{2cm}} \the\year~г.
\end{minipage}%
\bigskip

\settowidth{\ML}{«\underline{\hspace{0.6cm}}» \underline{\hspace{2cm}}}
\hfill\begin{minipage}{0.5\textwidth}
  Проверил\\
  \underline{\hspace{\ML}} П.\,Н.~Уланов\\
  «\underline{\hspace{0.7cm}}» \underline{\hspace{2cm}} \the\year~г.
\end{minipage}%
\vfill

\begin{center}
  Барнаул, \the\year~г.
\end{center}
\end{titlepage}

\tableofcontents

\newpage

\section{Введение и постановка задачи}


\section{Теоретическое описание задачи}


\section{Алгоритм и блок-схема}


\section{Проверка работы программы}


\addcontentsline{toc}{section}{Выводы по работе}
\section*{Выводы по работе}

\newpage

\addcontentsline{toc}{section}{Приложение}
\section*{Приложение}
\subsection*{Листинг main.py}
\addcontentsline{toc}{subsubsection}{Листинг main.py}
\inputminted[mathescape,linenos,breaklines]{python}{../src/main.py}

\subsection*{Листинг math.py}
\addcontentsline{toc}{subsubsection}{Листинг math.py}
\inputminted[mathescape,linenos,breaklines]{python}{../src/models/math.py}

\subsection*{Листинг argparser.py}
\addcontentsline{toc}{subsubsection}{Листинг argparser.py}
\inputminted[mathescape,linenos,breaklines]{python}{../src/utils/argparser.py}

\subsection*{Листинг input.py}
\addcontentsline{toc}{subsubsection}{Листинг input.py}
\inputminted[mathescape,linenos,breaklines]{python}{../src/utils/input.py}

\end{document}          
